\documentclass[11pt]{article}
\usepackage[pdftex]{graphicx}
\usepackage{amsfonts}
\oddsidemargin-.25in
\def\emph#1{{\em #1\/}}
\def\term#1{\emph{#1}}
\textwidth7in
\newcounter{thm}
\newtheorem{theorem}[thm]{Theorem}
\newtheorem{corollary}[thm]{Corollary}
\newtheorem{lemma}[thm]{Lemma}
\newtheorem{observation}[thm]{Observation}
\def\QED{$\checkmark$}
\def\ni{\noindent}
\def\tail#1{{\tau(#1)}}
\def\om#1{{\Omega(#1)}}
\newtheorem{obs}{Observation}
\def\Proof{\ni{\bf Proof:} }
\begin{document}
\title{Aperiodicity and the Curling Number Conjecture}
\author{D. A. Bailey\thanks{Contact.}\and D. Bonafilia\and T. Liu\and R. Poudyal\and A. Templeton\and D. Timilsina}
\maketitle

\begin{abstract}
The curling number, $cn(S)$, associated with a finite non-empty sequence of
digits, $S$ is the maximum $k$ such that $S$ can be written as $XY^k$ where $X$
is possibly empty and $Y$ is not.  The curling sequence is the iterative
expansion of $S_0=S$ by appending the curling number: $S_{i+1}=S_ic(S_i)$.  
The \term{curling number conjecture} suggests that this process ultimately
produces a string whose curling number is $1$.  We present new results on
\term{2,3-curling sequences}.  We demonstrate an interesting relationship
between the tails of curling sequences and words in an aperiodic grammar we
call the \term{A,B-grammar}.  We construct an \term{infinite curling sequence}
that appears consistent with \term{2,3-curling sequences} and prove a number
of interesting results.  We also report new data on the numbers of
\term{rotten sequences} up to length $48$.
\end{abstract}

\section{Introduction}
 Given a nonempty sequence $S$ it is always possible to write it as $XY^k$
where $Y$ is not empty.  The \term{curling number}, $cn(S)$, is the maximum
such $k$.  So, for example, the curling number associated with $2 2 3 2 3$ is $2$ by considering $X=2$ and $Y=2 3$, yielding $S=X\cdot Y\cdot Y$ (which we
write as $XY^2$).  By steps we can construct $S_0=S$ and build $S_{i+1}=S_i\cdot cn(S_i)$.  Thus $$S_1=2 2 3 2 3 2,~S_2=2 2 3 2 3 2 2,~S_3=2 2 3 2 3 2 2 2,$$ $$S_4=2 2 3 2 3 2 2 2 3,~S_5=2 2 3 2 3 2 2 2 3 1.$$

The \term{curling number conjecture} suggests that, given a finite start
sequence, this process ultimately constructs a string whose curling number is
1.  If a 1 would first be introduced after step $t$, we write the $t$-digit
tail, $T(s)=cn(S_0)cn(S_1)\cdots cn(S_{t-1})$.  In their study of the problem,
Chaffin et al.\cite{Ch13} were interested in characterizing $\tail{n}$, the maximum
length tail generated by an arbitrary starting sequence of length $n$.
A graph of $\tail{n}$ for relatively small values of $n$ is reproduced in
Figure~\ref{fig:tau}.  These values are certain for $n\le 48$ and ingenious
computational approaches were used to identify good candidates for $48<n\le 80$.

\begin{figure}[htbp]
\begin{center}
\includegraphics[width=4in]{tau.pdf}
\end{center}
\caption{A graph of $\tail{n}$ for $1\le n \le 80$ from \cite{Ch13}.}
\label{fig:tau}
\end{figure}

Given so little data, it is not clear what the growth pattern of $\tail{n}$ might be.  In an attempt to better understand sudden jumps $\tail{n}$, considerable
interest was given to \term{especially good starting sequences}, the unique sequences that led to new maximum values for tail length.  These sequences, again reproduced from \cite{Ch13}, are presented in Table~\ref{tab:egs}.

\begin{table}
\begin{center}
\begin{tabular}{|c|l|}
\hline
\bf n & Especially good starting sequence \\\hline
1&2\\
2&22\\
4&2323\\
6&222322\\
8&23222323\\
9&223222323\\
10&2323222322\\
11&22323222322\\
14&22323222322323\\
19&2232232322232232232\\
22&2322322323222323223223 \\
48&223223232223222322322232232322232223223222322323\\
68&22322322232232322232223223222322322232232322232223223222322322232232\\
76&2322232232223223232223222322322322232232223223232223222322322322232232223223\\
77&22322232322232223223222322232322232223223222322232322232223223232223223222323\\\hline
\end{tabular}
\end{center}
\caption{Especially good starting sequences, sequences that lead to new
maximum tail lengths, $\tail{n}$.  At their respective $n$, these long tail-producing sequences are unique and the starting sequences seem to meet several interesting properties. Sequences for $n>48$ are conjectured.}
\label{tab:egs}
\end{table}

Each of these sequences meets the following properties:
\begin{quote}
\begin{description}
\item[P2] Each sequence starts with a $2$.
\item[P3] No sequence contains adjacent $3$'s.
\item[P4] No sequence contains a nonempty string $V^4$.
\end{description}
\end{quote}
\ni We found these properties similar to properties associated with the study
of aperiodic sequences, and in particular, Gr\"unbaum's discussion of Conway's
\term{musical sequences}.  That got us to wondering: is there, perhaps, a
grammatical approach to thinking about the curling number conjecture?

In the next few sections, we describe an aperiodic system, what we call the
\term{curling grammar}, that describes the structure of (2,3)-curling sequences very
accurately, especially when the boundary conditions demanded by the problem's
finiteness are removed.  In Section~\ref{sect:ABg} we describe the grammar
and infinite curling sequences, which have surprising properties.  In
Section~\ref{sect:reflections} we consider the implications the curling
grammar has on our understanding of (2,3)-curling sequences.
In Section~\ref{sect:rotten} we update the search for doubly-rotten sequences.
We discuss progress we have made on understanding the open questions proposed
in \cite{Ch13} in Section~\ref{sect:results}.  Finally, we present our
conclusions in Section~\ref{sect:conclusions}.

\section{The AB-grammar}\label{sect:ABg}
We now present a new parallel string re-writing system (an \term{L-system}), the \term{AB-grammar}, that has many structural properties
that are similar to those found in the tails of starting curling sequences
composed of just 2's and 3's. Restricting the initial curling sequence to 
2's and 3's is, in a sense, the simplest problem since the tails of these
strings terminate immediately if a 1 is encountered or, within a single step
if a 4 is encountered.  Remarkably, even this restricted form of this problem
remains difficult to analyze.

As motivation for the following discussion, we present, here, the tail
of the especially good starting sequence for $n=8$:
$$S=2 3 2 2 2 3 2 3$$
\ni is extended to
$$S^{(e)}=2 3 2 \cdot 2 2 3 2 \cdot 3 2 2 2 3 2 2 2 3\cdot2 2 3 2\cdot2 2 3 2\cdot2 2 3 2\cdot3 2 2 2 3 2 2 2 3\cdot 2 2 3 2 \cdot 2 2 3 2\cdot 2 2 3 2\cdot 3 2 2 2 3 2 2 2 3\cdot 2 2 3 2\cdot 2 3 3 2$$
\ni We have inserted punctuation to delineate motifs that are interesting in this pattern.  Aside from the beginning and end ``boundary'' words, the entire sequence is constructed from two sequences:
$$A=2 2 3 2$$
$$B=3 2 2 2 3 2 2 2 3$$
\ni We can then think of $S^{(e)}$ in terms of these sequences:
$$S^{(e)}=232\cdot A \cdot B A A A B A A A B \cdot A \cdot 2332$$
With punctation added for emphasis.  We trust the reader will understand interest in the structural parallels between A-B-patterns and 2-3-patterns.
These parallels will be formalized in the next section.

\subsection{The AB-Grammar}

We now construct an L-system---a parallel string-rewriting system---based on
an alphabet of two symbols, $A$ and $B$, that contains two productions:
$$A\rightarrow A A B A$$
$$B\rightarrow B A A A B A A A B$$
\ni We call this system, the \term{curling grammar} or the \term{AB-grammar}.  The grammar starts with a string, $S$, and each symbol in the string
is rewritten, in parallel, to produce a new string.  Table~\ref{tab:deriv}
demonstrates the action of the AB-grammar on symbols $A$ and $B$.

\begin{table}
\begin{center}
\begin{tabular}{|c|c|}
\hline
Step & $A_i$ \\\hline
0 & $A$\\
1 & $AABA$\\
2 & $AABA\;AABA\;BAAABAAAB\;AABA$\\
 &\\
i+1 &  $AABA\;AABA\;B...B\;AABA$ \\
\hline
\end{tabular}\\[0.1in]
\begin{tabular}{|c|c|}
\hline
Step &$B$-derivation\\\hline
0 & $B$\\
1 & $BAAABAAAB$\\
2& $BAAABAAA\;AABA\;AABA\;AABA\;BAAABAAAB\;AABA\;AABA\;AABA\;BAAABAAAB$\\
&\\
i+1 & $BAAABAAAB\;A...A\;BAAABAAAB$\\
\hline
\end{tabular}
\end{center}
\caption{Action of the curling grammar on symbols $A$ (above) and $B$ (below).}

\label{tab:deriv}
\end{table}

While the strings we have seen so far are all finite length, the action of the
L-system is to generate strings through rewriting that are arbitrarily large
and, in the limit, the result is the existance of corresponding infinite
strings.  These strings can be rewritten, themselves, to produce other
infinite strings which are related by a process of \term{decomposition} (the
rewriting or \term{decomposing} of non-terminals into others) and its inverse,
\term{composition} (the consistent parsing or \term{composing} of letters to
form their anticedents).  The consistent composition of infinite strings is driven by the following observation.
\begin{observation}
The pattern $BAB$ uniquely corresponds to rewriting an $AB$ pattern,
while the pattern $BAAB$ uniquely corresponds to rewriting $BA$ patterns.
\end{observation}
\ni This knowledge allows us reverse the rewriting process:
replace the bracketed part of any instances of $BA\cdot BAAABAAAB\cdot AAB$
by a $B$ and then replace all instances of $AABA$ by $A$.

We can evaluate the growth characteristics of the rewriting rules with a
substitution matrix and characteristic equation:
$$Q=\left( \begin{array}{cc} 3 & 6 \\ 1 & 3 \end{array} \right),
\lambda^2-6\lambda +3=0$$ \ni (The left column of $Q$ describes the expansion
of $A$, the right column, $B$.) The eigenvalues of $Q$ are
$\lambda_1=3+\sqrt{6}\approx 5.45$ and $\lambda_2=3-\sqrt{6}\approx 0.55$.
The eigenvalue $\lambda_1$ corresponds to the steady-state growth rate
associated with this rewriting system.  Thus, we expect that words generated
by this grammar will generally expand by a bit more than $g=5.45$ on each
rewrite.  The corresponding eigenvector, $v_1=(\sqrt{6},1)$ describes relative
populations of $A$ and $B$ symbols (respectively), and we expect rewriting to
cause nontrivial strings to converge to having about $\rho=71\%$ $A$ symbols.
Again, these relative populations parallel what is seen in practice in curling
sequences: in our example above, 47 of 67 digits are $2$'s, about $70.15\%$.

What are the properties of infinite strings generated by this substitution?  Our first theorem describes many of the characteristics of infinite strings that are derived through AB-rewriting:
\def\scS{\mathcal{S}}
\def\scT{\mathcal{T}}
\begin{theorem}\label{thm:aperiodic}
Suppose $\scT$ is an infinite string resulting from 
consistent rewriting using the curling grammar.  Then the following properties are enjoyed by $\scT$:
\begin{enumerate}
\item (Non-periodicity) $\scT$ is not equivalent to itself under
non-trivial translation left or right.
\item (Local Isomorphism) Any finite substring, $S$, of $\scT$ appears
infinitely often with adjacent occurances appearing within a distance proportional to the length of $S$.
\end{enumerate}
\end{theorem}
\ni The proof of these properties are omitted here (see the Appendix), but they
follow the standard approaches of, say, Gr\"unbaum and Shephard\cite{Gr87}.
Thinking of $\scT$ as a tiling, then the parallel rewriting system is,
effectively, an inflation mechanism, that always results in non-periodic $t$.  We say the AB-tiling system is \term{aperiodic}.

To demonstrate the utility of the AB-grammar, we formally develop an analogy
between AB-tilings and infinite curling sequences.

\begin{theorem}
Suppose $\scT$ is an infinite AB-tiling.  If we replace all $A$ symbols with
2's and all $B$ symbols with 3's, then the resulting string $\scS$ can be
interpreted as a curling sequence.  That is, if $s$ is a digit
in $\scS$ then the sequence to the left of $s$ in $\scS$ can be written as
$XY^k$ (where $Y$ is finite, and $X$ is infinite) with $k$ attaining maximum
value $s$.
\end{theorem}
\ni The proof of this requires careful analysis of individual cases that
enumerate the different ways that $s$ was the result of rewriting. 
Proof under development.

\subsection{The Complementary AB-Grammar}
It is intriguing that the rewriting of $B$ in the AB-grammar is symmetric,
while the $A$ is not.  This led us to think about a complementary version of
the grammar, which we'll refer to as the A'B-Grammar, which has an alphabet of two
symbols, $A'$ and $B$, which rewrite in a manner opposite of the AB-grammar:
$$A'\rightarrow A' A' B A'$$
$$B\rightarrow B A' A' A' B A' A' A' B$$
\ni The finite strings generated by this language are simply the reversed
strings of the AB-grammar system.  In the A'B L-system, the occurance of the
$BAB$ sequence corresponds to the rewriting of $BA'$, while the occurance of
the $BAAB$ sequence correspons to the rewriting of $A'B$. We now have the
following surprising result.

\begin{theorem}
The infinite strings generated by the AB-grammar are the same as those generated by the A'B-grammar.
\end{theorem}
\ni This leads, rather obviously, to the following statement:
\begin{corollary}
Every infinite string generated by the AB-grammar can be parsed consistently
by either the AB-grammar or by the A'B-grammar.
\end{corollary}

A somewhat surprising result is our interpretation of similarly structured
curling sequences.
\begin{corollary}
Suppose $\scS$ is an 23-sequence derived from an infinite AB-tiling. Then
$\scS$ can also be interepreted as a curling sequence in reverse. 
That is, if $s$ is a digit
in $\scS$ then the sequence to the right of $s$ in $\scS$ can be written as
$Y^kX$ (where $Y$ is finite, and $X$ is infinite) with $k$ attaining maximum
value $s$.
\end{corollary}
\ni It is difficult to fully appreciate the impact of this result.  A little
inspection of the example at the top of this section validates the
interpretation of the curl in both directions.  It also suggests that, in a
finite string, extending the starting sequence is algorithmically related to
the development curls of the tail.

\subsection{The AB Neighborhoods}

\section{Reflections on 2,3-curling Sequences}\label{sect:reflections}

\section{Rotten Sequences}\label{sect:rotten}
\subsection{Computational Details}
\section{Results}\label{sect:results}
\section{Conclusions}\label{sect:conclusions}
\section*{Appendix: Proofs}\label{app:proofs}

\ni{\bf Proof of Theorem~\ref{thm:aperiodic}.} (motivated by Gr\"unbaum, p. 573, 10.6.3).
An infinite AB-sequence (a \term{curly sequence}) cannot be written as infinitely many repetitions of
a finite block of A's and B's.

Suppose it were the case—--that is an infinite AB-sequence $\scS$ can be
written as a finite sequence $S$ of length $m$, repeated.  Then, since $\scS$
is an curly sequence its terms can be composed consistently using the
AB-grammar, forming a precursor sequence, $\scS'$ (which is also curly) with
each letter of $\scS^{-1}$ expanding to 4 (in the case of an $A$) or 9 (in the
case of a $B$) letters in $\scS$.  We may, of course do this any finite number
of times, say, $n$, yielding $\scS^{-n}$ whose letters that span more than
$4^n$ letters in $\scS$.  If $n$ is sufficiently large, then $m < 4^n$.  But
shifting by $m$---a supposed symmetry of $\scS$ does not yield a symmetry in
$\scS^{-n}$, so we have that $\scS$ is non-periodic.\QED

\ni{\bf Alternative proof.}(Modeled after Senechal\cite{Se95})  
We remind the reader that the
rewriting matrix for the AB grammar is \term{primitive}.  A primitive matrix
$A$ is one whose entries are non-negative with, for some $k$, the entries of
$A^k$ are all positive. (For our rewriting grammar, $k=1$.)  Such a matrix
demonstrates that ``all letters eventually appear'' in the rewriting of any
patch of letters.  It is sufficient, then to consider the rewriting of the
letter $A$.  Here, $A\rightarrow A A B A \rightarrow A A B A B A A A B A A A B
\rightarrow \cdots$.  At each stage we can count the number of letters that
appear, and we can compute the ratio of $A$'s and $B$'s.  As we demonstrated
in the text the ratio of $A$'s to $B$'s is $\sqrt{6}$ to $1$.  Since
$\sqrt{6}$ is irrational the rewriting system does not generate an infinite
periodic sequence.  Why? If the sequence had been periodic, there would be
rational ratio of $A$'s and $B$'s in each block, preserved through simple
repetition.\QED

\medskip
\ni(Local Isomorphism.)  Here, we seek to prove that every finite subsequence, $S$, of letters of a curly sequence appears in infinitely often elsewhere in the sequence.

Here, we again note that every infinite curly sequence is the result of
rewriting a precursor sequence $\scS^{-1}$, itself infinite and curly.
Considering precursors some finite number of steps---say $n$---back, we can encapsulate
the finite subsequence in the image a single letter.  All letters, of course,
appear infinitely often, so in their images the finite subsequence must also
appear infinitely often.  Furthermore, it is possible to identify a bound
on $n$ as a function of the diameter of the patch.  This, along with the fact
that no letter is farther than 4 from an equivalent identifies an upperbound
on the distance that must be traveled in $\scS$ to find a patch congruent with
$S$.  We have, then, demonstrated the Local Isomorphism property.\QED

\newpage
\pagestyle{empty}
\nocite{Ch09,Ch13,VDB07}
\bibliographystyle{plain}
\bibliography{references}
\end{document}

Thoughts on property P2.

Suppose we have an especially good starting sequence, S.  Then S begins
with a 2.

Proof:
  Suppose, instead, that it began with a 3.  This 3 is not neutral because
if it was neutral, then we could remove the 3 and the shorter sequence would
have a tail, just as long, contradicting the fact that this is the first
sequence that has a tail so long.
  Since the 3 is not neutral, it is the beginning of a Y that supports a
curling number, c, at some point in the tail, lc+1, where |Y|=l.  
Of course, c could be 2 or three.
  Suppose it is 2.  Then we have this diagram
  3...3...2... = SS(e)
  Y-->Y-->2...
  ^   ^   ^
  1   l+1 2l+1


  Suppose c=3.  Then we have this diagram
  3...3...3...3... = SS(e)
  Y-->Y-->Y-->3...
  ^   ^   ^   \
  1   l+1 2l+1 3l+1
Now let Y' be Y with the leftmost 3 rotated to be rightmost
(e.g. Y=323222 becomes Y'=232223) then we know SS(e) begins
  3Y'Y'Y'...
which means that the curl at 3l+2 is 3, which places 2 3's in a row which
(besides violating proposed property P3) terminates the sequence.
[Why this might be bad: the position 3l+1 seems quite early for a long tail to end.]

[Corollary: the Y that proves a curl of 3 does not start with 3.]

