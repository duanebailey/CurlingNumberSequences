\documentclass[11pt]{article}
\usepackage{amsfonts}
\oddsidemargin-.25in
\def\emph#1{{\em #1\/}}
\def\term#1{\emph{#1}}
\textwidth7in
\def\QED{$\checkmark$}
\def\ni{\noindent}
\def\tail#1{{\tau(#1)}}
\def\om#1{{\Omega(#1)}}
\newtheorem{obs}{Observation}
\def\Proof{\ni{\bf Proof:} }
\begin{document}
%% \begin{center}
%% {\bf Thoughts on Rottenness}\\
%% Duane Bailey, Tong Liu\\
%% Williams College Department of Computer Science\\
%% \end{center}
\def\cn#1{\mathrm{cn}(#1)}
\def\tail#1{\tau(#1)}
\def\Sext{S^{(e)}}
We consider, here, thoughts on rotten sequences.

\bigskip

\ni{\bf Notation.}  We will follow the conventions of Chaffin, et al. and use the following notation.  Sequences will be denoted by uppercase letters, for example, $S$, $X$ and $Y$.  As
with formal language theory, $Y^k$ is the sequence derived by concatenating
$k$ copies of $Y$.  We will use $|S|$ to indicate the length of $S$.  The
curling number associated with a sequence $S$ is written $\cn{S}$ and is taken
to be the maximum $k$ such that $S=XY^k$.  Suppose $S_0=s_1s_2\cdots s_l$, then
we will typically define $S_{i+1}=S_{i}\cdot\cn{S_i}$, the extension of $S_i$ by
concatenating its curling number.  The Curling Number Conjecture is that
for some smallest (finite) $n$, $\cn{S_n}=1$.  When this is true, we define
$\Sext=S_n=s_1s_2\cdots s_{l+n}$ and $\tail{S_0}=s_{l+1}s_{l+2}\cdots s_{l+n}$.
Notice that the these strings do not include the final~1.

\section{Initial Observations}
Suppose we have a sequence $S$ with tail $T=\tail{S}$.  Computing the tails of starting
sequences that are prefixed with $2$ or $3$ yields $S_2=2S$ with tail $T_2=\tail{2S}$
or $S_3=3S$ with tail $T_3=\tail{3S}$, respectively.  If $|T_2| < |T|$ or $|T_3|<|T|$
we say that $S$ is \term{2-rotten} or \term{3-rotten}, respectively.  A sequence
that is both 2- and 3-rotten we say is \term{doubly rotten}.  To date, no doubly rotten sequences have been found for $|S|\le 34$.  In part, this article
discusses our approach to extending these results to $|S|=48$.

Is there a way, given $S$ and $T$ to quickly identify if $S$ is 2- or 3-rotten?
Let's assume, without loss of generality, that $S$ is 2-rotten.  Then, we have
that $|T_2|<|T|$ and, in particular, these strings must differ first in some
digit, $i$, relative to the beginning of the tail.  How might these differ? Adding a 2 at the front causes the difference, it must be the case that digit $i$ of $T$ increases---by one---in $T_2$.
Thus $T_2$ is lexagraphically larger than $T$, which we'll write informally as $T_2>T$.
Suppose that digit $i$ of $T_2$ is $k$.  Then we have that
$$2S\cdot T_2=Y^kZ_2$$
\ni and
$$ST=XY^{k-1}Z$$
\ni with $Y=2X$ and where $Z_2$ is the suffix, after position $i$, of $T_2$ and $Z$ is the suffix of $T$ with $|Z_2|<|Z|$. Furthermore, we know that $i=(1+(k|Y|))-|2S|$,
or $|Y|=(i+|2S|-1)/k$.  Thus $i$ in $S_2$ is equivalent to $1$ mod $k$.

\section{Computing Curl Values} 
We can leverage these observations to identify the onset of rottenness as
we compute the tail, $T$ of $S$.  Let's think about how that might work.

First, we develop a 2-dimensional table, $M[i,j]$.  $M[0,0]=0$.  We index sequences starting
with character $1$.  The first row, $M[0,j]$, is $S_j$, the $j$th character
of the string $S$.  Row $i>0$, at entry $j$, is the number of times that 
$S_j$ repeats, stepping backward with a stride of $i$ characters at a time.  (Note that $M[i,j]$, $j\le i$ is $1$.)  To determine the curl of the first $j$ characters of $S$,
we compute for each row $i>0$, the minimum of the last $i$ entries of the 
$i$th row of $M$, that is the $\mathrm{min}_{k=1}^nM[i,j-i+k]$.  Effectively, we're looking to see how often the $i$-digit
suffix of $S$ ($s_{j-i+1}s_{j-i+2}\cdots s_j$) repeats.  The curl is, then the maximum value of $C_i$, $0<i<=\frac{j}{2}$.  The bound of $\frac{j}{2}$ derives from the fact that a curl of
2 can only happen if $Y$ has length no more than half the length of $S$.

\ni{\bf Example.} Suppose we start with the 8 character string, $S=23222323$.
\begin{center}
\begin{tabular}{|c||c|c|c|c|c|c|c|c||c|c|}
\hline
$M[i\downarrow,j\rightarrow]$ & 1 & 2 & 3 & 4 & 5 & 6 & 7 & 8 & 9 & 10 \\\hline\hline
0 &  2 & 3 & 2 & 2 & 2 & 3 & 2 & 3 & {\em 2} & 2\\\hline
1 &  1 & 1 & 1 & 2 & 3 & 1 & 1 & {\em 1} & 1 &\\
2 &  1 & 1 & 2 & 1 & 3 & 1 & {\em 4} & {\em 2} & 5 &\\
3 &  1 & 1 & 1 & 2 & 1 & {\em 1} & {\em 3} & {\em 1} & 1&\\
4 &  1 & 1 & 1 & 1 & {\em 2} & {\em 2} & {\em 2} & {\em 1} & 3&\\
5 &  1 & 1 & 1 & 1 & 1 & 1 & 1 & 1 & 2&\\
6 &  1 & 1 & 1 & 1 & 1 & 1 & 2 & 2 & 2&\\
7 &  1 & 1 & 1 & 1 & 1 & 1 & 1 & 1 & 1&\\
8 &  1 & 1 & 1 & 1 & 1 & 1 & 1 & 1 & 2&\\\hline
\end{tabular}
\end{center}
\ni The computation of $M[0,9]$, the curl of $S[1..8]$ computes, for each row
$1\le i\le 4$, the minimum of each of the final $i$ values (shown in italics). The curl is the maximum of these computed minima.

This process continues as far as is required with at each step, a new row added,
and a new column of repeat-factors is computed and, finally, the curl computation to head the next column.

\section{Detecting Rottenness}
How can we detect potential rottenness?  A character in the tail of the
string leads to rottenness when prefixing a 2 or 3 causes a curl to increase
by one.\footnote{Notice that the value of the curl would never increase from
1 to 2 since that would, by definition, lengthen the tail.}  This fact
\emph{may} lead to a tail that is shorter.

As motiviation, let's look at the computation of $\Sext{S}$ for $S=32323$ a
2-rotten sequence:
\begin{center}
\begin{tabular}{|c||c|c|c|c|c||c|c|c|c|c|}
\hline
$M[i\downarrow,j\rightarrow]$ & 1 & 2 & 3 & 4 & 5 & 6 & 7 & 8 & 9 & 10 \\\hline\hline
0 &  3 & 2 & 3 & 2 & 3 & 2 & 3 & {\em 3} & 2 & 1 \\\hline
1 &  1 & 1 & 1 & 1 & 1 & 1 & 1 & 2 & 1 & \\ 
2 &  1 & 1 & 2 & 2 & 3 & {\em 3} & 4 & 1 & 1 & \\ 
3 &  1 & 1 & 1 & 1 & 1 & 1 & 1 & 2 & 2 & \\ 
4 &  1 & 1 & 1 & 1 & 2 & 2 & 2 & 1 & 1 & \\ 
5 &  1 & 1 & 1 & 1 & 1 & 1 & 1 & 2 & 2 & \\ 
6 &  1 & 1 & 1 & 1 & 1 & 1 & 1 & 1 & 1 & \\ 
7 &  1 & 1 & 1 & 1 & 1 & 1 & 1 & 1 & 2 & \\ 
8 &  1 & 1 & 1 & 1 & 1 & 1 & 1 & 1 & 1 & \\ \hline
\end{tabular}
\end{center}

\end{document}
