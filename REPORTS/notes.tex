\documentclass[11pt]{article}
\usepackage{amsfonts}
\oddsidemargin-.25in
\def\emph#1{{\em #1\/}}
\def\term#1{\emph{#1}}
\textwidth7in
\def\QED{$\checkmark$}
\def\ni{\noindent}
\def\tail#1{{\tau(#1)}}
\def\om#1{{\Omega(#1)}}
\newtheorem{obs}{Observation}
\def\Proof{\ni{\bf Proof:} }
\begin{document}
\begin{center}
{\bf New Thoughts on the Curling Sequence Conjecture}\\
Duane Bailey, Tong Liu, Diwas Timilsina\\
Williams College Department of Computer Science\\
\end{center}

This note collects a number of observations and theorems we have developed in the study
of the Curling Sequence Conjecture.  It is assumed that the reader is familiar with 
previous work on this topic.

We will assume that we are investigating, as is the case in the original paper, starting
sequences that contain 2's and 3's.  Given this, we will refer to the individual entries
in the sequence as \term{digits}.  We say that a starting sequence of $n$ digits is \term{good} if it
generates the longest tail generated by any $n$ digit starting sequence.  We say that $n$ is a
jump point if the length of the longest tail $\tail{S}$ from a starting sequence, $S$, of $n$ characters is longer
than any tail based on $n-1$ characters.  In these notes we assume that the length
of these longest tails, $\om{n}$, is nondecreasing.  It is also known that, for jump point $n$, that the
good starting sequence is unique.  We refer to these as \term{especially good starting sequences}.

We begin with a number of observations that have not been stated before.

\begin{obs}Every especially good starting sequence at jump point $n$, $S=s_1s_2...s_n$, $s_i\in\{2,3\}$, ends with a digit $s_n$ that does not, in fact,
represent the curl of $S^-=s_1s_2...s_{n-1}$.
\end{obs}

\Proof A proof by contradiction.  Suppose $s_n$ was the curl of
$s_1...s_{n-1}$.  Then, we note that $S^-$ is a $n-1$ digit sequence that
generates a tail of length $\tail{S}+1$: the beginning of the tail is $s_n$
and the subsequent digits are simply those that form the tail of
$S$.  Since $S$ was a good starting sequence at a jump point $n$, the extistance
of a tail of length $\om{n}+1$ from a starting sequence of $n-1$ digits is (under our main conjecture) a 
contradiction.  It follows, then, that the last digit of the jump point's good
starting sequence was not, in fact, a valid curl.\QED

Let's call this digit that is not the correct curl a \term{flaw}.  Since starting
sequences contain only 2's and 3's, it is clear, then, that $s_1$ and $s_2$ are flawed, 
as well.  There may be others.

Generally, we will use $s_i$ to refer to digits of the starting sequence, $t_i$ to refer to digits of a tail, and $a_i$ to refer to digits, generically.

We say that a digit in a sequence \term{supports} a later curl $a_i=k$, $k>1$,
if that digit necessarily appears in repeating component in the calculation of
$a_i$.  We say any digit that does not support a curl as \term{neutral}.  It
is straightforward to note that any neutral digits must appear at the
beginning of the sequence.

\begin{obs} Especially good starting sequences contain no neutral digits.
\end{obs}
\Proof Suppose $S=s_1...s_n$ is the especially good starting sequence at jump point $n$.  Thus $\om{n-1}<\om{n}$.  It has tail of length
$\tail{S}$.  We can now construct $S^-=s_2...s_n$, a starting sequence of $n-1$ digits that generates
a tail of $\tail{S}$ digits (if it didn't generate this tail, the $s_1$ supported the computation of
some curl in the tail, a contradiction to its being neutral). But, we have then, that 
$\tail{S^-}=\tail{S}=\om{n}$, but this violates that conjectured requirement that $\om{n-1}<\om{n}$.\QED

We call any starting sequence with a neutral digit, \term{weak}.

Because there are no neutral digits in a good starting sequence, some
non-trivial digit in the tail has a $Y^k$ that includes all the flaws.
Obviously, if these flawed digits are copied (as part of the starting
sequence) to a location in the tail, they are not flawed in their new
locations.

There are three important properties that have been observed for good
starting sequences:
\begin{description}
\item[P2.]  All good starting sequences start with digit 2.
\item[P3.]  No good starting sequence contains consecutive 3 digits.
\item[P4.]  No good starting sequence contains a substring $V^4$.
\end{description}
\ni Property~4 particularly eliminates $2222$.  Notice properties P3 and P4
do not necessarily hold in the tails.

Past work focused on identifying especially good starting sequences between
$n=49$ and $n=80$, based on investigating only starting sequences that met
each of the three properties.  Since the properties are not known to be true
of all especially good starting sequences, it would be nice to know that these
are true, as a means to simplify searches and proofs.

What can we say about P3?  Suppose that an especially good starting sequence
contained a pair of adjacent 3's.  We note that if the tail is longer than the 
especially good starting sequence (and it appears to be for all known n), then
the starting sequence appears as part of the tail.  Why?
First, we note that the sequence starts with $s_1$ which is not neutral.  Furthermore, this character supports some curl in the tail.  (Why? Suppose it only
supported a curl in the starting sequence.  It could, then be removed, shortening the starting sequence to $n-1$ digits, with a similar tail.  But this would
violate our assumptions about the nondecreasing growth of $\om{n}$.)  There
are two cases: either $ST=s_1Xs_1Xt_iY$ where $t_i=2$ or $ST=s_1Xs_1Xs_1Xt_iY$, where $t_i=3$.

\begin{obs}
Suppose a curl of $3$ is supported by the first digit of a sequence.  The first
digit must be $2$.
\end{obs}
\Proof Suppose we know that a curl is 3 at some point in the tail, and that it
is supported by $s_1$.  Then we can write $ST=s_1Xs_1Xs_1X3Y$, where $|X|=l-1$.
Then we know that $s_1$ logically appears at postitions $1$, $l+1$, and $2l+1$.
Assume that $s_1$ is 3.  Then we can write the first $2l$ digits as $s_1Xs_1X=AB^3$.  If $|A|=0$, then some prefix to position $3l+1$ can be written as
$B^4$ which generates a curl of 4, terminates, and violates our assumption that the curl at $3l$ is 3. If $|A|>0$, then $ST=3CAB^33CAB^33CAB^33Y$ but then the curl at $3l+1$ is supported by the first character of the third instance of $B^3$, and not $s_1$. It follows that $s_1$ must be 2.  (Could $2l\le |S_0|$ and thus,
perhaps, flawed? See, for example the $t_3$ entry of $n=6. Can we show that there is \term{another} curl supported by $s_1$, later?)

Suppose a curl of $2$ is supported by the first digit of the starting sequence.
Can $s_1$ be 3?  Suppose it is.  Then we have the whole sequence 
starting as $ST=3X3X2Y$, where the 2 falls at position $i$ of the tail or
$2l+1$ of the whole sequence.  If $|S|\le l$, then the $3$ at location $l+1$
would suggest that $3X$ has curl $3$, but that contradicts the fact that
there is a curl of $2$ after $3X3X$.  If $|S| > l$ then ?

\begin{obs} (Diwas) Extensions of especially good starting sequences start
with 2.
\end{obs}

\end{document}
