\documentclass[11pt]{article}
\usepackage{amsfonts}
\oddsidemargin-.25in
\def\emph#1{{\em #1\/}}
\def\term#1{\emph{#1}}
\textwidth7in
\def\QED{$\checkmark$}
\def\ni{\noindent}
\def\tail#1{{\tau(#1)}}
\def\om#1{{\Omega(#1)}}
\newtheorem{obs}{Observation}
\def\Proof{\ni{\bf Proof:} }
\begin{document}
\begin{center}
{\bf Thoughts on Rottenness}\\
Duane Bailey, Tong Liu\\
Williams College Department of Computer Science\\
\end{center}

We consider, here, thoughts on rotten sequences.

Suppose we have a sequence $S$ with tail $T$.  Computing the tails of starting
sequences that are prefixed with $2$ or $3$ yields $S_2=2S$ with tail $T_2$
or $S_3=3S$ with tail $T_3$, respectively.  If $|T_2| < |T|$ or $|T_3|<|T|$
we say that $S$ is \term{2-rotten} or \term{3-rotten}, respectively.  A sequence
that is 2- and 3-rotten we say is \term{doubly rotten}.  To date, no doubly
rotten sequences have been found for $|S|<35$.

Is there a way, given $S$ and $T$ to quickly identify if $S$ is 2- or 3-rotten?
Let's assume, without loss of generality, that $S$ is 2-rotten.  Then, we have
that $|T_2|<|T|$ and, in particular, these strings must differ first in some
digit, $i$.  How might these differ?  It has to be the case that $T_2$ is
lexagraphically larger that $T$, which we'll write informally as $T_2>T$.
Suppose that digit $i$ of $T_2$ is $k$.  Then we have that
$$2ST_2=Y^kZ_2$$
\ni and
$$ST=XY^{k-1}Z$$
\ni where $Y=2X$ and $|Z_2|<|Z|$.  Furthermore, we know that $i=(k|Y|)-|2S|$,
or $|Y|=\frac{(i+|2S|)}k$.  Thus the offset in $S_2$ is a multiple of $k$.



\end{document}
